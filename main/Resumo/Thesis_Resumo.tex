%%%%%%%%%%%%%%%%%%%%%%%%%%%%%%%%%%%%%%%%%%%%%%%%%%%%%%%%%%%%%%%%%%%%%%%%
%                                                                      %
%     File: Thesis_Resumo.tex                                          %
%     Tex Master: Thesis.tex                                           %
%                                                                      %
%     Author: Andre C. Marta                                           %
%     Last modified :  2 Jul 2015                                      %
%                                                                      %
%%%%%%%%%%%%%%%%%%%%%%%%%%%%%%%%%%%%%%%%%%%%%%%%%%%%%%%%%%%%%%%%%%%%%%%%

\section*{Resumo}

% Add entry in the table of contents as section
\addcontentsline{toc}{section}{Resumo}

O presente relatório foi elaborado no âmbito do estágio curricular do Mestrado Integrado em Medicina Veterinária da Universidade de Évora, podendo ser dividido em duas componentes principais. A primeira visa apresentar um resumo da casuística acompanhada pelo autor ao longo de seis meses de estágio, decorridos entre 2 de setembro e 28 de fevereiro, no OneVet- Hospital Veterinário do Porto.

A segunda componente corresponde a uma monografia subordinada ao tema “Sialoadenose responsiva a fenobarbital em cães”, incluindo ainda a descrição de um caso clínico acompanhado durante o referido estágio.

A \textit{sialoadenose} responsiva a fenobarbital é uma patologia rara, caracterizada por um aumento bilateral e indolor do parênquima das glândulas salivares, sem envolvimento inflamatório ou neoplásico. Supõe-se que os sinais clínicos estejam associados a uma forma de epilepsia límbica, sendo o tratamento médico baseado em fármacos antiepiléticos, como o fenobarbital. 

\vfill

\textbf{\Large Palavras-chave:} palavra-chave1, palavra-chave2

