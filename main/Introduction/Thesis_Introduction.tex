%%%%%%%%%%%%%%%%%%%%%%%%%%%%%%%%%%%%%%%%%%%%%%%%%%%%%%%%%%%%%%%%%%%%%%%%
%                                                                      %
%     File: Thesis_Introduction.tex                                    %
%     Tex Master: Thesis.tex                                           %
%                                                                      %
%     Author: Andre C. Marta                                           %
%     Last modified :  2 Jul 2015                                      %
%                                                                      %
%%%%%%%%%%%%%%%%%%%%%%%%%%%%%%%%%%%%%%%%%%%%%%%%%%%%%%%%%%%%%%%%%%%%%%%%

\chapter{Introduction}
\label{chapter:introduction}

O presente relatório foi desenvolvido no âmbito do estágio curricular do Mestrado Integrado em Medicina Veterinária da Universidade de Évora, realizado entre 2 de setembro e 28 de fevereiro de 2025 no OneVet- Hospital Veterinário do Porto (HVP), sob orientação interna da Doutora Teresa Oliveira e orientação externa da Dra. Nina Rodrigues.

Este relatório encontra-se dividido em duas componentes:
\begin{enumerate}
    \item A primeira componente consiste numa análise descritiva da casuística acompanhada pelo autor durante o período de estágio, organizada pelas áreas de medicina preventiva, clínica cirúrgica, clínica médica, incluindo ainda uma breve referência aos exames complementares de diagnóstico.
    \item A segunda parte é dedicada à monografia sobre o tema “Sialoadenose responsiva a fenobarbital em cães”, complementada pela descrição de um caso clínico acompanhado no decurso do estágio.
\end{enumerate}






