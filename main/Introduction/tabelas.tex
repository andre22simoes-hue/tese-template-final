\begin{table}
\centering
\begin{tblr}{
  width = \linewidth,
  colspec = {Q[325]Q[190]Q[177]Q[140]Q[100]},
  cell{2}{2} = {c},
  cell{2}{3} = {c},
  cell{2}{4} = {c},
  cell{2}{5} = {c},
  cell{3}{2} = {c},
  cell{3}{3} = {c},
  cell{3}{4} = {c},
  cell{3}{5} = {c},
  cell{4}{2} = {c},
  cell{4}{3} = {c},
  cell{4}{4} = {c},
  cell{4}{5} = {c},
  cell{5}{2} = {c},
  cell{5}{3} = {c},
  cell{5}{4} = {c},
  cell{5}{5} = {c},
  hlines,
  vlines,
}
\textbf{Procedimentos}   & \textbf{Fip (Canina)} & \textbf{Fip (Felina)} & \textbf{Fi (Total)} & \textbf{Fr (\%)} \\
Vacinação                & 11                    & 6                     & 17                  & \textbf{54.84}   \\
Desparasitação           & 6                     & 7                     & 13                  & \textbf{41.94}   \\
Identificação eletrónica & 2                     & 2                     & 4                   & \textbf{12.90}   \\
Total                    & \textbf{19}           & \textbf{12}           & \textbf{31}         & \textbf{100}     
\end{tblr}
\end{table}


% \usepackage{tabularray}
\begin{table}
\centering
\begin{tblr}{
  width = \linewidth,
  colspec = {Q[121]Q[165]Q[229]Q[123]Q[113]Q[102]Q[79]},
  cells = {c},
  cell{1}{1} = {c=3}{0.515\linewidth},
  cell{2}{1} = {r=5}{},
  cell{2}{2} = {r=3}{},
  cell{5}{2} = {c=2}{0.394\linewidth},
  cell{6}{2} = {c=2}{0.394\linewidth},
  cell{7}{1} = {c=3}{0.515\linewidth},
  cell{8}{1} = {c=3}{0.515\linewidth},
  cell{9}{1} = {c=3}{0.515\linewidth},
  cell{10}{1} = {c=3}{0.515\linewidth},
  cell{11}{1} = {c=3}{0.515\linewidth},
  cell{12}{1} = {c=2,r=3}{0.286\linewidth},
  cell{15}{1} = {c=2,r=2}{0.286\linewidth},
  cell{17}{1} = {c=3}{0.515\linewidth},
  vlines,
  hline{1-2,7-12,15,17-18} = {-}{},
  hline{3-4,13-14,16} = {3-7}{},
  hline{5-6} = {2-7}{},
}
\textbf{Exames Complementares deDiagnóstico} &                  &                        & \textbf{Fi (Canina)} & \textbf{Fi (Felina)} & \textbf{Fi (Total)} & \textbf{Fr (\%)} \\
Imagiologia                                  & Ultrassonografia & Abdominal              & 156                  & 187                  & 343                 & \textbf{52.3}    \\
                                             &                  & Ecocardiografia        & 44                   & 27                   & 71                  & \textbf{10.8}    \\
                                             &                  & Ocular                 & 2                    & 0                    & 2                   & \textbf{0.3}     \\
                                             & Radiografia      &                        & 41                   & 35                   & 76                  & \textbf{11.6}    \\
                                             & TC               &                        & 26                   & 5                    & 29                  & \textbf{4.4}     \\
Eletrocardiografia                           &                  &                        & 15                   & 12                   & 27                  & \textbf{4.1}     \\
Colheita de líquido Cefalorraquidiano        &                  &                        & 4                    & 2                    & 6                   & \textbf{0.9}     \\
Urinálise                                    &                  &                        & 16                   & 25                   & 41                  & \textbf{6.3}     \\
Teste de Schirmer                            &                  &                        & 2                    & 0                    & 2                   & \textbf{0.3}     \\
Teste de fluoresceína                        &                  &                        & 1                    & 0                    & 1                   & \textbf{0.2}     \\
Testes rápidos                               &                  & Parvovirus/Coronavirus & 7                    & 5                    & 12                  & \textbf{1.8}     \\
                                             &                  & SNAP PLi               & 1                    & 3                    & 4                   & \textbf{0.6}     \\
                                             &                  & FIV/FELV               & 0                    & 4                    & 4                   & \textbf{0.6}     \\
Anatomo-histopatologia                       &                  & Citologia              & 11                   & 18                   & 29                  & \textbf{4.4}     \\
                                             &                  & Biópsia                & 4                    & 3                    & 7                   & \textbf{1.1}     \\
\textbf{TOTAL}
                             &                  &                        & 330                  & 326                  & 656                 & 100              
\end{tblr}
\caption{O andre e gay} 
\label{tab:andre}
\end{table}

sfsdfgsdfguydsgsdgfugf tabel \ref{tab:andre}
% \usepackage{tabularray}
\begin{table}
\centering
\begin{tblr}{
  width = \linewidth,
  colspec = {Q[492]Q[133]Q[121]Q[108]Q[83]},
  cells = {c},
  hlines,
  vlines,
}
\textbf{Cardiologia}                         & \textbf{Fi (Canina)} & \textbf{Fi (Felina)} & \textbf{Fi (Total)} & \textbf{Fr (\%)} \\
Doença Degenerativa da Válvula Mitral (DDVM) & 27                   & 0                    & 27                  & \textbf{39.13}   \\
Cardiomiopatia hipertrófica                  & 0                    & 12                   & 12                  & \textbf{17.39}   \\
Cardiomiopatia dilatada                      & 9                    & 0                    & 9                   & \textbf{13.04}   \\
Bloqueio atrioventricular                    & 3                    & 0                    & 3                   & \textbf{4.35}    \\
Tromboembolismo aórtico                      & 0                    & 3                    & 3                   & \textbf{~~ 4.35} \\
Estenose sub-aórtica                         & 3                    & 0                    & 3                   & \textbf{4.35}    \\
Falso tendão                                 & 3                    & 0                    & 3                   & \textbf{4.35}    \\
Movimento anterior sistólico (SAM)           & 0                    & 3                    & 3                   & \textbf{4.35}    \\
Persistência ducto arterioso                 & 2                    & 0                    & 2                   & \textbf{2.90}    \\
Estenose pulmonar                            & 0                    & 1                    & 1                   & \textbf{1.45}    \\
Insuficiência da válvula tricúspide          & 1                    & 0                    & 1                   & \textbf{1.45}    \\
Comunicação interventricular                 & 0                    & 1                    & 1                   & \textbf{1.45}    \\
Cor tiatriatum                               & 1                    & 0                    & 1                   & \textbf{1.45}    \\
\textbf{TOTAL}                               & 49                   & 20                   & 69                  & \textbf{100}     
\end{tblr}
\end{table}

% \usepackage{tabularray}
\begin{table}
\centering
\begin{tblr}{
  width = \linewidth,
  colspec = {Q[396]Q[162]Q[148]Q[133]Q[94]},
  cells = {c},
  hlines,
  vlines,
}
\textbf{Neurologia}            & \textbf{Fi (Canina)} & \textbf{Fi (Felina)} & \textbf{Fi (Total)} & \textbf{Fr (\%)} \\
Hérnia discal                  & 8                    & 0                    & 8                   & \textbf{34.78}   \\
Epilepsia idiopática           & 5                    & 0                    & 5                   & \textbf{21.74}   \\
Meningoencefalite              & 5                    & 0                    & 5                   & \textbf{21.74}   \\
Síndrome vestibular geriátrico & 2                    & 0                    & 2                   & \textbf{8.70}    \\
Discoespondilite               & 1                    & 0                    & 1                   & \textbf{4.35}    \\
Malformação chiari like        & 1                    & 0                    & 1                   & \textbf{4.35}    \\
Miastenia gravis               & 1                    & 0                    & 1                   & \textbf{4.35}    \\
\textbf{TOTAL}                 & 23                   & 0                    & 23                  & 100              
\end{tblr}
\end{table}

% \usepackage{tabularray}
\begin{table}
\centering
\begin{tblr}{
  width = \linewidth,
  colspec = {Q[469]Q[144]Q[135]Q[108]Q[83]},
  cells = {c},
  hlines,
  vlines,
}
\textbf{Oncologia}                             & \textbf{Fip (Canina)} & \textbf{Fip (Felina)} & \textbf{Fi (Total)} & \textbf{Fr (\%)} \\
Linfoma
  alimentar                            & 0                     & 15                    & 15                  & \textbf{27.78}   \\
Carcinoma
  das células escamosas              & 0                     & 7                     & 7                   & \textbf{12.96}   \\
Hemangiossarcoma
  esplénico                   & 6                     & 0                     & 6                   & \textbf{11.11}   \\
Mastocitoma
  cutâneo                          & 6                     & 0                     & 6                   & \textbf{11.11}   \\
Carcinoma
  mamário                            & 1                     & 4                     & 5                   & \textbf{9.26}    \\
Carcinoma
  das células de transição da bexiga & 2                     & 0                     & 2                   & \textbf{3.70}    \\
Linfoma
  mediastínico                         & 0                     & 2                     & 2                   & \textbf{3.70}    \\
Insulinoma                                     & 2                     & 0                     & 2                   & \textbf{3.70}    \\
Melanoma
  oral                                & 2                     & 0                     & 2                   & \textbf{3.70}    \\
Adenocarcinoma
  das glândulas hepatóides      & 1                     & 0                     & 1                   & \textbf{1.85}    \\
Adenoma
  esplénico                            & 1                     & 0                     & 1                   & \textbf{1.85}    \\
Adenoma
  hepático                             & 1                     & 0                     & 1                   & \textbf{1.85}    \\
Carcinoma pulmão                               & 1                     & 0                     & 1                   & \textbf{1.85}    \\
Hemangiossarcoma
  cardíaco                    & 1                     & 0                     & 1                   & \textbf{1.85}    \\
Hepatocarcinoma                                & 1                     & 0                     & 1                   & \textbf{1.85}    \\
Leucemia                                       & 1                     & 0                     & 1                   & \textbf{1.85}    \\
Linfoma
  gástrico                             & 1                     & 0                     & 1                   & \textbf{1.85}    \\
Linfoma ocular                                 & 0                     & 1                     & 1                   & \textbf{1.85}    \\
Melanocitoma
  palpebral                       & 1                     & 0                     & 1                   & \textbf{1.85}    \\
Osteossarcoma                                  & 1                     & 0                     & 1                   & \textbf{1.85}    \\
Timoma                                         & 1                     & 0                     & 1                   & \textbf{1.85}    \\
\textbf{TOTAL}                                 & 30                    & 28                    & 54                  & 100              
\end{tblr}
\end{table}

% \usepackage{tabularray}
\begin{table}
\centering
\begin{tblr}{
  width = \linewidth,
  colspec = {Q[475]Q[138]Q[127]Q[113]Q[85]},
  cells = {c},
  hlines,
  vlines,
}
\textbf{Endocrinologia}                   & \textbf{Fi (Canina)} & \textbf{Fi (Felina)} & \textbf{Fi (Total)} & \textbf{Fr (\%)} \\
Diabetes \textit{mellitus}                & 4                    & 10                   & 14                  & \textbf{42.42}   \\
Hipertiroidismo                           & 0                    & 8                    & 8                   & \textbf{24.24}   \\
Hiperadrenocorticismo hipófise-dependente & 4                    & 0                    & 4                   & \textbf{12.12}   \\
Hipoadrenocorticismo                      & 3                    & 0                    & 4                   & \textbf{12.12}   \\
Hipotiroidismo                            & 3                    & 0                    & 4                   & \textbf{12.12}   \\
Diabetes insipidus                        & 1                    & 0                    & 1                   & \textbf{3.03}    \\
\textbf{TOTAL}                            & 15                   & 18                   & 33                  & \textbf{100}     
\end{tblr}
\end{table}

% \usepackage{tabularray}
\begin{table}
\centering
\begin{tblr}{
  width = \linewidth,
  colspec = {Q[527]Q[121]Q[112]Q[100]Q[79]},
  cells = {c},
  hlines,
  vlines,
}
\textbf{Oftalmologia}                                             & \textbf{Fi (Canina)} & \textbf{Fi (Felina)} & \textbf{Fi (Total)} & \textbf{Fr (\%)} \\
Queratoconjuntivite seca                                          & 2                    & 0                    & 2                   & \textbf{18.18}   \\
Úlcera superficial da córnea                                      & 2                    & 0                    & 2                   & \textbf{18.18}   \\
Cataratas                                                         & 1                    & 0                    & 1                   & \textbf{9.09}    \\
Descemetocélio                                                    & 1                    & 0                    & 1                   & \textbf{9.09}    \\
Entrópio                                                          & 1                    & 0                    & 1                   & \textbf{9.09}    \\
Ectrópio                                                          & 1                    & 0                    & 1                   & \textbf{9.09}    \\
Prolapso da glândula da membrana nictitante (\textit{cherry eye}) & 1                    & 0                    & 1                   & \textbf{9.09}    \\
Síndrome de Degeneração Retiniana Adquirida Súbita                & 1                    & 0                    & 1                   & \textbf{9.09}    \\
Uveíte                                                            & 1                    & 0                    & 1                   & \textbf{9.09}    \\
\textbf{TOTAL}                                                    & 11                   & 0                    & 11                  & 100              
\end{tblr}
\end{table}

% \usepackage{tabularray}
\begin{table}
\centering
\begin{tblr}{
  width = \linewidth,
  colspec = {Q[465]Q[140]Q[131]Q[115]Q[85]},
  cells = {c},
  hlines,
  vlines,
}
\textbf{Doenças infeciosas e parasitárias} & \textbf{Fi (Canina)} & \textbf{Fi (Felina)} & \textbf{Fi (Total)} & \textbf{Fr (\%)} \\
Parvovirose canina/ Panleucopenia felina   & 5                    & 6                    & 11                  & \textbf{28.95}   \\
Leucemia felina (FELV)                     & 0                    & 5                    & 5                   & \textbf{13.16}   \\
Imunodeficiência viral felina (FIV)        & 0                    & 4                    & 4                   & \textbf{10.53}   \\
Leishmaniose                               & 4                    & 0                    & 4                   & \textbf{10.53}   \\
Peritonite infeciosa felina                & 0                    & 4                    & 4                   & \textbf{10.53}   \\
Calicivirose                               & 0                    & 3                    & 3                   & \textbf{~7.89}   \\
Leptospirose                               & 3                    & 0                    & 3                   & \textbf{7.89}    \\
Coriza                                     & 0                    & 2                    & 2                   & \textbf{5.26}    \\
Coronavírus                                & 1                    & 0                    & 1                   & \textbf{2.63}    \\
Dirofilariose                              & 1                    & 0                    & 1                   & \textbf{2.63}    \\
\textbf{TOTAL}                             & 14                   & 24                   & 38                  & \textbf{100}     
\end{tblr}
\end{table}

% \usepackage{tabularray}
\begin{table}
\centering
\begin{tblr}{
  width = \linewidth,
  colspec = {Q[477]Q[142]Q[133]Q[106]Q[81]},
  cells = {c},
  hlines,
  vlines,
}
\textbf{Gastroenterologia e gl. anexas}            & \textbf{Fip (Canina)} & \textbf{Fip (Felina)} & \textbf{Fi (Total)} & \textbf{Fr (\%)} \\
Pancreatite                                        & 6                     & 15                    & 21                  & \textbf{20.39}   \\
Doença
  inflamatória intestinal                   & 1                     & 13                    & 14                  & \textbf{13.59}   \\
Obstrução
  por corpo estranho gástrico/intestinal & 11                    & 3                     & 14                  & \textbf{13.59}   \\
Gastroenterite
  inespecífica                      & 7                     & 0                     & 7                   & \textbf{6.80}    \\
Enteropatia
  com perda de proteína                & 6                     & 0                     & 6                   & \textbf{5.83}    \\
Gastroenterite
  hemorrágica                       & 6                     & 0                     & 6                   & \textbf{5.83}    \\
Hepatite
  crónica                                 & 6                     & 0                     & 6                   & \textbf{5.83}    \\
Colangio-hepatite                                  & 1                     & 4                     & 5                   & \textbf{4.85}    \\
Fecaloma                                           & 3                     & 0                     & 3                   & \textbf{2.91}    \\
Fístula perianal                                   & 3                     & 0                     & 3                   & \textbf{2.91}    \\
Lipidose
  hepática                                & 0                     & 3                     & 3                   & \textbf{2.91}    \\
Dilatação
  torção gástrica                        & 2                     & 0                     & 2                   & \textbf{1.94}    \\
Mucocelo                                           & 2                     & 0                     & 2                   & \textbf{1.94}    \\
Peritonite sética                                  & 0                     & 2                     & 2                   & \textbf{1.94}    \\
Triadite felina                                    & 0                     & 2                     & 2                   & \textbf{1.94}    \\
Cirrose
  hepática                                 & 1                     & 0                     & 1                   & \textbf{0.97}    \\
Colecistite                                        & 1                     & 0                     & 1                   & \textbf{0.97}    \\
Megacólon
  idiopático                             & 4                     & 1                     & 1                   & \textbf{0.97}    \\
Sialoadenose                                       & 1                     & 0                     & 1                   & \textbf{0.97}    \\
\textbf{TOTAL}                                     & 60                    & 43                    & 103                 & \textbf{100}     
\end{tblr}
\end{table}

% \usepackage{tabularray}
\begin{table}
\centering
\begin{tblr}{
  width = \linewidth,
  colspec = {Q[425]Q[154]Q[142]Q[127]Q[90]},
  cells = {c},
  hlines,
  vlines,
}
\textbf{Hematologia}           & \textbf{Fi (Canina)} & \textbf{Fi (Felina)} & \textbf{Fi (Total)} & \textbf{Fr (\%)} \\
Anemia hemolitica imunomediada & 4                    & 1                    & 5                   & \textbf{83.33}   \\
Trombocitopénia imunomediada   & 1                    & 0                    & 1                   & \textbf{16.67}   \\
\textbf{TOTAL}                 & 5                    & 1                    & 6                   & 100              
\end{tblr}
\end{table}

% \usepackage{tabularray}
\begin{table}
\centering
\begin{tblr}{
  width = \linewidth,
  colspec = {Q[408]Q[160]Q[146]Q[131]Q[92]},
  cells = {c},
  hlines,
  vlines,
}
\textbf{Nefrologia e urologia}     & \textbf{Fi (Canina)} & \textbf{Fi (Felina)} & \textbf{Fi (Total)} & \textbf{Fr (\%)} \\
Doença renal crónica (DRC)         & 7                    & 30                   & 37                  & \textbf{52.11}   \\
Cistite Idiopática Felina          & 0                    & 7                    & 7                   & \textbf{9.86}    \\
Cistolitiase                       & 4                    & 2                    & 6                   & \textbf{8.45}    \\
Ureterolitíase                     & 0                    & 6                    & 6                   & \textbf{8.45}    \\
Uretrolitíase                      & 0                    & 5                    & 5                   & \textbf{7.04}    \\
Infeção do trato urinário inferior & 0                    & 3                    & 3                   & \textbf{4.23}    \\
Doença renal Poliquística          & 0                    & 2                    & 2                   & \textbf{2.82}    \\
Hidronefrose                       & 2                    & 0                    & 2                   & \textbf{2.82}    \\
Lesão renal aguda (LRA)\textbf{}   & 0                    & 1                    & 1                   & \textbf{1.41}    \\
Megaureter                         & 0                    & 1                    & 1                   & \textbf{1.41}    \\
Ureter ectópico                    & 0                    & 1                    & 1                   & \textbf{1.41}    \\
\textbf{TOTAL}                     & 13                   & 58                   & 71                  & 100              
\end{tblr}
\end{table}

% \usepackage{tabularray}
\begin{table}
\centering
\begin{tblr}{
  width = \linewidth,
  colspec = {Q[438]Q[150]Q[138]Q[123]Q[88]},
  cells = {c},
  hlines,
  vlines,
}
\textbf{Otorrinolaringologia}                       & \textbf{Fi (Canina)} & \textbf{Fi (Felina)} & \textbf{Fi (Total)} & \textbf{Fr (\%)} \\
Otite externa por \textbf{\textit{Malassezia spp.}} & 7                    & 0                    & 7                   & \textbf{38.89}   \\
Otite externa bacteriana                            & 6                    & 0                    & 6                   & \textbf{33.33}   \\
Oto-hematoma                                        & 2                    & 1                    & 3                   & \textbf{16.67}   \\
Otite média bacteriana                              & 1                    & 0                    & 1                   & \textbf{5.56}    \\
Otite média por pólipo nasofaríngeo                 & 0                    & 1                    & 1                   & \textbf{5.56}    \\
\textbf{TOTAL}                                      & 16                   & 2                    & 18                  & 100              
\end{tblr}
\end{table}

% \usepackage{tabularray}
\begin{table}
\centering
\begin{tblr}{
  width = \linewidth,
  colspec = {Q[463]Q[142]Q[131]Q[115]Q[85]},
  cells = {c},
  hlines,
  vlines,
}
\textbf{Pneumologia}                  & \textbf{Fi (Canina)} & \textbf{Fi (Felina)} & \textbf{Fi (Total)} & \textbf{Fr (\%)} \\
Colapso traqueal                      & 6                    & 0                    & 6                   & \textbf{28.57}   \\
Pneumonia aspirativa                  & 4                    & 0                    & 4                   & \textbf{19.05}   \\
Asma felina                           & 0                    & 3                    & 3                   & \textbf{14.29}   \\
Bronquite crónica                     & 3                    & 0                    & 3                   & \textbf{14.29}   \\
Contusão pulmonar secundária a trauma & 2                    & 0                    & 2                   & \textbf{9.52}    \\
Quilotórax idiopático                 & 0                    & 2                    & 2                   & \textbf{9.52}    \\
Hérnia Diafragmática                  & 1                    & 0                    & 1                   & \textbf{4.76}    \\
\textbf{TOTAL}                        & 16                   & 5                    & 21                  & 100              
\end{tblr}
\end{table}
% \usepackage{tabularray}
\begin{table}
\centering
\begin{tblr}{
  width = \linewidth,
  colspec = {Q[279]Q[194]Q[179]Q[158]Q[113]},
  cells = {c},
  hlines,
  vlines,
}
\textbf{Teriogeneologia} & \textbf{Fi (Canina)} & \textbf{Fi (Felina)} & \textbf{Fi (Total)} & \textbf{Fr (\%)} \\
Quisto prostático        & 4                    & 0                    & 4                   & \textbf{50.00}   \\
Piómetra                 & 3                    & 0                    & 3                   & \textbf{37.50}   \\
Parafimose               & 1                    & 0                    & 1                   & \textbf{12.50}   \\
Prostatite               & 1                    & 0                    & 1                   & \textbf{12.50}   \\
Vaginite                 & 1                    & 0                    & 1                   & \textbf{12.50}   \\
\textbf{TOTAL}           & 8                    & 0                    & 8                   & 100              
\end{tblr}
\end{table}
% \usepackage{tabularray}
% \usepackage{tabularray}
\begin{table}
\centering
\begin{tblr}{
  width = \linewidth,
  colspec = {Q[502]Q[129]Q[119]Q[106]Q[81]},
  cells = {c},
  hlines,
  vlines,
}
\textbf{Traumatologia e Doenças Musculoesqueléticas} & \textbf{Fi (Canina)} & \textbf{Fi (Felina)} & \textbf{Fi (Total)} & \textbf{Fr (\%)} \\
Rotura do ligamento cruzado cranial                  & 4                    & 0                    & 4                   & \textbf{50.0}    \\
Fraturas ósseas                                      & 1                    & 2                    & 3                   & \textbf{37.5}    \\
Poliartrite
  imunomediada                           & 1                    & 0                    & 1                   & \textbf{12.5}    \\
\textbf{TOTAL}                                       & 6                    & 2                    & 8                   & 100              
\end{tblr}
\end{table}

% \usepackage{tabularray}
\begin{table}
\centering
\begin{tblr}{
  width = \linewidth,
  colspec = {Q[431]Q[152]Q[140]Q[125]Q[88]},
  cells = {c},
  hlines,
  vlines,
}
\textbf{Toxicologia}                & \textbf{Fi (Canina)} & \textbf{Fi (Felina)} & \textbf{Fi (Total)} & \textbf{Fr (\%)} \\
Rodenticidas                        & 3                    & 0                    & 3                   & \textbf{60.0}    \\
Intoxicação por lagarta do pinheiro & 1                    & 0                    & 1                   & \textbf{20.0}    \\
Paracetamol                         & 1                    & 0                    & 1                   & \textbf{20.0}    \\
\textbf{TOTAL}                      & 5                    & 0                    & 5                   & 100              
\end{tblr}
\end{table}

% \usepackage{tabularray}
\begin{table}
\centering
\begin{tblr}{
  width = \linewidth,
  colspec = {Q[471]Q[140]Q[129]Q[113]Q[85]},
  cells = {c},
  hlines,
  vlines,
}
\textbf{Dermatologia }                     & \textbf{Fi (Canina)} & \textbf{Fi (Felina)} & \textbf{Fi (Total)} & \textbf{Fr (\%)} \\
Dermatite atópica/ Síndrome atópica felina & 15                   & 2                    & 17                  & \textbf{68.00}   \\
Alopécia X                                 & 3                    & 0                    & 3                   & \textbf{12.00}   \\
Pioderma superficial                       & 3                    & 0                    & 3                   & \textbf{12.00}   \\
Pênfigo foliáceo                           & 1                    & 1                    & 2                   & \textbf{8.00}    \\
\textbf{TOTAL}                             & 22                   & 3                    & 25                  & 100              
\end{tblr}
\end{table}

% \usepackage{tabularray}
\begin{table}
\centering
\begin{tblr}{
  width = \linewidth,
  colspec = {Q[452]Q[150]Q[140]Q[112]Q[83]},
  cells = {c},
  hlines,
  vlines,
}
\textbf{Tecidos moles}                           & \textbf{Fip (Canina)} & \textbf{Fip (Felina)} & \textbf{Fi (Total)} & \textbf{Fr (\%)} \\
Orquiectomia                                     & 4                     & 6                     & 10                  & \textbf{27.78}   \\
Ovário-histerectomia (OVH )                      & 5                     & 3                     & 8                   & \textbf{22.22}   \\
Enterotomia                                      & 5                     & 1                     & 6                   & \textbf{16.67}   \\
Colocação de tubo de esofagostomia               & 0                     & 4                     & 4                   & \textbf{11.11}   \\
Colocação de\textit{ bypass} ureteral subcutâneo & 0                     & 3                     & 3                   & \textbf{8.33}    \\
Gastrotomia                                      & 3                     & 0                     & 3                   & \textbf{~ 8.33}  \\
Mastectomia                                      & 2                     & 1                     & 3                   & \textbf{8.33}    \\
Nodulectomia                                     & 3                     & 0                     & 3                   & \textbf{8.33}    \\
Colocação pacemaker                              & 2                     & 0                     & 2                   & \textbf{5.56}    \\
Resolução de oto-hematoma                        & 2                     & 0                     & 2                   & \textbf{5.56}    \\
Biópsia pâncreas                                 & 1                     & 0                     & 1                   & \textbf{2.78}    \\
Enterectomia                                     & 0                     & 1                     & 1                   & \textbf{2.78}    \\
\textbf{TOTAL}                                   & 27                    & 19                    & 36                  & 100              
\end{tblr}
\end{table}

% \usepackage{tabularray}
\begin{table}
\centering
\begin{tblr}{
  width = \linewidth,
  colspec = {Q[454]Q[146]Q[135]Q[119]Q[87]},
  cells = {c},
  hlines,
  vlines,
}
\textbf{Odontologia}                   & \textbf{Fi (Canina)} & \textbf{Fi (Felina)} & \textbf{Fi (Total)} & \textbf{Fr (\%)} \\
Destartarização e/ou extração dentária & 8                    & 2                    & 10                  & \textbf{90.91}   \\
Remoção de melanoma oral               & 1                    & 0                    & 1                   & \textbf{9.09}    \\
\textbf{TOTAL}                         & 9                    & 2                    & 11                  & 100              
\end{tblr}
\end{table}

% \usepackage{tabularray}
\begin{table}
\centering
\begin{tblr}{
  width = \linewidth,
  colspec = {Q[281]Q[202]Q[188]Q[150]Q[106]},
  cells = {c},
  hlines,
  vlines,
}
\textbf{Neurocirurgia} & \textbf{Fip (Canina)} & \textbf{Fip (Felina)} & \textbf{Fi (Total)} & \textbf{Fr (\%)} \\
Hemilaminectomia       & 1                     & 0                     & 1                   & \textbf{ 50 }    \\
Ventral slot           & 1                     & 0                     & 1                   & \textbf{ 50 }    \\
TOTAL                  & 2                     & 0                     & 2                   & 100              
\end{tblr}
\end{table}

% \usepackage{tabularray}
\begin{table}
\centering
\begin{tblr}{
  width = \linewidth,
  colspec = {Q[488]Q[138]Q[129]Q[104]Q[81]},
  cells = {c},
  hlines,
  vlines,
}
\textbf{Cirurgia Ortopédica~}                        & \textbf{Fip (Canina)} & \textbf{Fip (Felina)} & \textbf{Fi (Total)} & \textbf{Fr (\%)} \\
Pins para resolução de múltiplas fraturas de dígitos & 0                     & 1                     & 1                   & \textbf{ 100 }   \\
\textbf{ TOTAL }                                     & 0                     & 1                     & 1                   & 100              
\end{tblr}
\end{table}

% \usepackage{tabularray}
\begin{table}
\centering
\begin{tblr}{
  width = \linewidth,
  colspec = {Q[348]Q[175]Q[162]Q[142]Q[102]},
  cells = {c},
  hlines,
  vlines,
}
\textbf{Outros procedimentos} & \textbf{Fi (Canina)} & \textbf{Fi (Felina)} & \textbf{Fi (Total)} & \textbf{Fr (\%)} \\
Cistocentese                  & 31                   & 46                   & 77                  & \textbf{50.0}    \\
Toracocentese                 & 13                   & 11                   & 24                  & \textbf{15.6}    \\
Limpeza de feridas            & 11                   & 7                    & 18                  & \textbf{11.7}    \\
Transfusão sanguínea          & 5                    & 6                    & ~11                 & \textbf{7.1}     \\
Eutanásia                     & 6                    & 4                    & 10                  & \textbf{6.5}     \\
Quimioterapia                 & 4                    & 6                    & 10                  & \textbf{6.5}     \\
Abdominocentese               & 4                    & 0                    & 4~~                 & \textbf{2.6}     \\
\textbf{TOTAL}                & 74                   & 80                   & 154                 & 100              
\end{tblr}
\end{table}