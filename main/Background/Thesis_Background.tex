%%%%%%%%%%%%%%%%%%%%%%%%%%%%%%%%%%%%%%%%%%%%%%%%%%%%%%%%%%%%%%%%%%%%%%%%
%                                                                      %
%     File: Thesis_Background.tex                                      %
%     Tex Master: Thesis.tex                                           %
%                                                                      %
%     Author: Andre C. Marta                                           %
%     Last modified :  2 Jul 2015                                      %
%                                                                      %
%%%%%%%%%%%%%%%%%%%%%%%%%%%%%%%%%%%%%%%%%%%%%%%%%%%%%%%%%%%%%%%%%%%%%%%%

\chapter{Casuística}
\label{chapter:Casuistica}



\section{Descrição do local de estágio}


O OneVet-Hospital Veterinário do Porto (HVP), localizado na cidade do Porto, possui mais de 20 anos de experiência na área de Medicina Veterinária. A sua equipa está disponível 24 horas por dia, ao longo de todo o ano, capacitada para atender qualquer emergência ou intervenção cirúrgica com a maior celeridade. 

Este hospital conta com uma equipa multidisciplinar altamente qualificada em diversas áreas, sendo considerado uma referência em Cardiologia, Dermatologia, Oncologia, Ortopedia, Traumatologia e Oftalmologia. Dispõe ainda de uma vasta gama de meios complementares de diagnóstico, incluindo Ecografia, Ecocardiografia, Radiografia, Tomografia Computorizada, Fluoroscopia, Endoscopia e Ressonância Magnética.

A constante dedicação e especialização da equipa contribui significativamente para o reconhecimento do HVP como hospital de referência, resultando assim, numa elevada casuística.
Durante o período de estágio, o hospital mudou de instalações, o que permitiu melhorar as condições de trabalho e aprendizagem, bem como o bem-estar e conforto dos doentes. 

O novo espaço apresenta uma receção comum, com zonas de espera separadas para cães e gatos, de forma a minimizar o stresse. Dispõe de seis consultórios (três para cães e três para gatos), equipados com difusores de feromonas e de uma sala dedicada a momentos mais delicados, como a eutanásia, garantindo a privacidade dos tutores.

O HVP conta ainda uma área destinada a doenças infetocontagiosas, com separação entre cães e gatos, uma unidade de cuidados intensivos, quatro áreas de internamento (cães de grande porte, restantes cães, gatos e animais exóticos), sala para preparação cirúrgica e duas salas cirúrgicas devidamente equipadas. O serviço de Imagiologia inclui todos os meios referidos anteriormente, dispondo ainda de laboratório de análises clínicas, farmácia, sala de esterilização e uma sala de quimioterapia.

\section{Análise de Casuística}

A estrutura de estágio baseou-se numa rotação semanal pelas diversas áreas: “Internamento”, “Consultas”, “Oncologia”, “Cardiologia”,” Dermatologia”, “Medicina Interna” e “Cirurgia”. Em todos os serviços, foi promovida a participação ativa dos estagiários nos procedimentos clínicos, sempre sob supervisão de médicos e/ou enfermeiros.

Foram realizados turnos no Internamento, em horário variável (das 8h00 às 16h00 e das 13h00 às 21h00), no serviço de Consultas e Cardiologia (das 12h00 às 20h00), em Cirurgia (das 9h00 às 17h00) e em Oncologia, Medicina Interna e Dermatologia (das 9h00 às 17h00). Adicionalmente, efetuaram-se turnos noturnos semanais (das 20h00 às 8h00), durante os quais foi possível acompanhar consultas de urgência e respetivas abordagens.

Numa primeira fase, será apresentada uma análise estatística da casuística acompanhada pelo autor durante os seis meses de estágio. Esta análise baseia-se nos casos observados durante os turnos diurnos e noturnos dos diversos serviços do hospital.

É importante salientar que a casuística aqui descrita não reflete a totalidade dos casos hospitalares, mas apenas os efetivamente acompanhados pelo autor. Adicionalmente, muitos doentes apresentavam doenças concomitantes, pelo que o número total de casos registados poderá não corresponder ao número real de animais acompanhados.

De forma a facilitar a análise, a casuística foi dividida em três áreas clínicas: a Medicina Preventiva, Clínica Médica e Clínica Cirúrgica que foram, por sua vez, sub-categorizadas para permitir uma melhor análise destas áreas:

\begin{enumerate}
    \item Medicina preventiva: vacinação, desparasitação e Identificação eletrónica.
    \item 	Clínica médica: Cardiologia,Hematologia, Dermatologia, Doenças Infeciosas e Parasitárias, Endocrinologia, Gastroenterologia e Glândulas Anexas, Teriogenologia, Nefrologia e Urologia, Neurologia, Oftalmologia, Oncologia, Traumatologia e Doenças Músculoesqueléticas, Otorrinolaringologia, Pneumologia e Toxicologia. 
    \item Clínica cirúrgica: Ortopedia, Neurocirurgia, Tecidos Moles, Odontologia e Outros Procedimentos.
\end{enumerate}

Os casos foram organizados através da utilização de tabelas que incluem a frequência absoluta por espécie (Fip), exclusivamente animais da espécie canina e felina, a frequência absoluta do procedimento ou doença (Fi) e a frequência relativa em percentagem (Fr (\%)).

\section{Distribuição da Casuística por espécie animal}

A distribuição casuística abrange as espécies canina (\textit{Canis lupus familiaris}) e felina (\textit{Felis catus}). Ao longo do estágio, foram registados, pelo autor, um total de $n=565$ casos referentes a estas espécies. A espécie canina apresentou maior representatividade, com uma frequência absoluta de $n=339$, enquanto a espécie felina correspondeu a $n=226$. Em termos de frequência relativa(Fr), estes valores equivalem a 60,0\% e 40,0\%, respetivamente. Não foram incluídos animais pertencentes a espécies exóticas, dado que a Médica Veterinária responsável por esta área se encontrava ausente por motivo de baixa médica. 


\begin{figure}[h!]
  \centering
  \begin{tikzpicture}
    \pie[color={uered, uegray}]{60/Canídeo, 40/Felino}
  \end{tikzpicture}
  \caption{Frequência relativa em percentagem (Fr (\%)), das espécies contempladas na 
casuística (N=565)}
  \label{fig:circular}
\end{figure}


\section{Distribuição da Casuística por área clínica}

A distribuição da casuística por área clínica, apresentada na tabela \ref{tab:t1} , compreende um total de $n=588$ casos observados, divididos pelas três principais áreas da clínica de animais de companhia: "Clínica Médica", "Clínica Cirúrgica" e "Medicina Preventiva". Verificando os dados, a área com maior representatividade foi a Clínica Médica, com uma frequência relativa de $84\%$, seguida da Clínica Cirúrgica, com  $10,2\%$ e, por último, a Medicina Preventiva, com uma Fr inferior ($5,8\%$). 
Conforme referido anteriormente, o número de animais observados é inferior ao número total de casos clínicos registados, devido, o que se justifica pela presença de doenças concomitantes e pela existência de situações que necessitaram de seguimento cirúrgico.

% \usepackage{tabularray}
\begin{table}[h!]
\caption{Distribuição da casuística pelas três principais áreas clínicas, por espécie (Fip), frequência absoluta (Fi) e frequência relativa (Fr(\%))} 
 \label{tab:t1}
\centering
\begin{tblr}{
  width = \linewidth,
  colspec = {Q[300]Q[196]Q[183]Q[146]Q[104]},
  cells = {c},
  hlines,
  vlines,
}
\textbf{Área Clínica} & \textbf{Fip (Canina)} & \textbf{Fip (Felina)} & \textbf{Fi (Total)} & \textbf{Fr (\%)} \\
Clínica Médica        & 289                   & 205                   & 494                 & \textbf{84,0}    \\
Clínica Cirúrgica     & 38                    & 22                    & 60                  & \textbf{10,2}    \\
Medicina Preventiva   & 19                    & 15                    & 34                  & \textbf{5,8}     \\
\textbf{ Total }      & \textbf{346}          & \textbf{242}          & \textbf{588}        & \textbf{100}     
\end{tblr}
\end{table}

\subsection{Medicina Preventiva}

Como referido anteriormente, a área clínica Medicina Preventiva foi aquela em que se registou o menor número de casos. Uma das principais razões para esta menor casuística relaciona-se com o reduzido contacto do autor com esta área específica durante o período de estágio. O procedimento mais representado foi a Vacinação com $n=17$ casos, correspondente a uma frequência relativa de ($50,0\%$), seguido da Desparasitação com $n=13$ casos ($38,2\%$), e por último a Identificação Eletrónica com $n=4$ casos ($11,8\%$). 

% \usepackage{tabularray}
\begin{table}[h!]
\caption{Distribuição da casuística de procedimentos observados na área clínica Medicina Preventiva, por espécie (Fip), frequência absoluta (Fi) e frequência relativa (Fr(\%))} 
\label{tab:t2}
\centering
\begin{tblr}{
  width = \linewidth,
  colspec = {Q[325]Q[190]Q[177]Q[140]Q[100]},
  cells = {c},
  hlines,
  vlines,
}
\textbf{Procedimentos}   & \textbf{Fip (Canina)} & \textbf{Fip (Felina)} & \textbf{Fi (Total)} & \textbf{Fr (\%)} \\
Vacinação                & 11                    & 6                     & 17                  & \textbf{50,0}    \\
Desparasitação           & 6                     & 7                     & 13                  & \textbf{38,2}    \\
Identificação Eletrónica & 2                     & 2                     & 4                   & \textbf{11,8}    \\
\textbf{ Total }         & 19                    & 15                    & 34                  & \textbf{100}     
\end{tblr}
\end{table}

\subsection{Clínica Médica}

A Clínica Médica representa a área clínica que apresentou maior casuística com um total de $n=494$ casos, correspondendo a uma frequência relativa de $84,0\%$. Trata-se de uma área que representa todas as doenças diagnosticadas durante os seis meses de estágio, subdivido em 15 áreas, posteriormente organizadas por ordem decrescente em relação ao número de casos.

Nesta área, o autor desenvolveu várias atividades desde a contenção de animais, colheita de sangue, procedimento de venóclise, preparação de materiais para a realização de procedimentos, medição de pressões arteriais, preparação e administração de medicações, Toracocentese e Abdominocentese, e até auxílio dos Médicos e Enfermeiros Veterinários em diversas funções, como exames imagiológicos complementares, sendo eles a Ultrassonografia, Raio-X e Tomografia Computorizada.

Pode-se então averiguar pela tabela \ref{tab:t3}, que a área com maior representatividade foi Gastroenterologia e Glândulas Anexas, com um total de $n=103$ casos ($Fr=20,8\%$), sendo, por esse motivo, a primeira a ser descrita no âmbito da Clínica Médica. No extremo oposto, temos a Toxicologia, com apenas $n=5$ casos ($Fr=1,0\%$).

  % \usepackage{tabularray}
\begin{table}[h!]
\caption{Distribuição da casuística recolhida em Clínica Médica, por espécie (Fip), por frequência 
absoluta (Fi) e frequência relativa em percentagem (Fr(\%))} 
\label{tab:t3}
\centering
\begin{tblr}{
  width = \linewidth,
  colspec = {Q[473]Q[142]Q[133]Q[106]Q[81]},
  cells = {c},
  hlines,
  vlines,
}
\textbf{Área Clínica}                       & \textbf{Fip (Canina)} & \textbf{Fip (Felina)} & \textbf{Fi (Total)} & \textbf{Fr (\%)} \\
Gastroenterologia e Glândulas Anexas                      & 60                    & 43                    & 103                 & 20,8             \\
Nefrologia e Urologia                       & 13                    & 58                    & 71                  & 14,4             \\
Cardiologia                                 & 49                    & 20                    & 69                  & 14,0             \\
Oncologia                                   & 24                    & 29                    & 53                  & 10,7             \\
Doenças Infeciosas e Parasitárias           & 14                    & 24                    & 38                  & 7,7              \\
Endocrinologia                              & 15                    & 18                    & 33                  & 6,7              \\
Dermatologia                                & 22                    & 3                     & 25                  & 5,1              \\
Neurologia                                  & 23                    & 0                     & 23                  & 4,7              \\
Pneumologia                                 & 16                    & 5                     & 21                  & 4,3              \\
Otorrinolaringologia                        & 16                    & 2                     & 18                  & 3,6              \\
Oftalmologia                                & 11                    & 0                     & 11                  & 2,2              \\
Teriogenologia                              & 10                    & 0                     & 10                  & 2,0              \\
Traumatologia e Doenças Musculoesqueléticas & 6                     & 2                     & 8                   & 1,6              \\
Hematologia                                 & 5                     & 1                     & 6                   & 1,2              \\
Toxicologia                                 & 5                     & 0                     & 5                   & 1,0              \\
\textbf{TOTAL}                              & 289                   & 205                   & 494                 & 100              
\end{tblr}
\end{table}
\subsubsection{Gastroenterologia e Glândulas Anexas}

A especialidade de Gastroenterologia e Glândulas Anexas dedica-se ao tratamento do trato digestivo, bem como de órgãos e Glândulas Anexas como o Pâncreas, Fígado, vesícula biliar e vias biliares. Esta foi a área, dentro da Clínica Médica, que mais casos registou. Foi responsável por $17,5\%$ da casuística recolhida durante o estágio curricular, correspondendo a $20,8\%$ de casos recolhidos em Clínica Médica.

De acordo com a tabela \ref{tab:t4} a afeção com maior representatividade foi a Pancreatite, com  $n=21$ casos($Fr=20,4\%$) recolhidos. Em contraste, com uma frequência relativa de apenas $1,0\%$, doenças como Cirrose Hepática, Colecistite e Sialoadenose, foram as menos representadadas.

% \usepackage{tabularray}
\begin{table}[h!]
\caption{Distribuição da casuística recolhida na especialidade Gastroenterologia e Glândulas Anexas, por 
espécie (Fip), por frequência absoluta (Fi), e frequência relativa em percentagem (Fr (\%)). } 
\label{tab:t4}
\centering
\begin{tblr}{
  width = \linewidth,
  colspec = {Q[477]Q[142]Q[133]Q[106]Q[81]},
  cells = {c},
  hlines,
  vlines,
}
\textbf{Gastroenterologia e gl. anexas}            & \textbf{Fip (Canina)} & \textbf{Fip (Felina)} & \textbf{Fi (Total)} & \textbf{Fr (\%)} \\
Pancreatite                                        & 6                     & 15                    & 21                  & \textbf{20,4}    \\
Doença
  Inflamatória Intestinal                   & 1                     & 13                    & 14                  & \textbf{13,6}    \\
Obstrução
  por Corpo Estranho Gástrico/Intestinal & 1                     & 3                     & 14                  & \textbf{13,6}    \\
Gastroenterite
  Inespecífica                      & 7                     & 0                     & 7                   & \textbf{6,8}     \\
Enteropatia
  com Perda de Proteína                & 6                     & 0                     & 6                   & \textbf{5,8}     \\
Gastroenterite
  Hemorrágica                       & 6                     & 0                     & 6                   & \textbf{5,8}     \\
Hepatite
  Crónica                                 & 6                     & 0                     & 6                   & \textbf{5,8}     \\
Colangio-Hepatite                                  & 1                     & 4                     & 5                   & \textbf{4,9}     \\
Megacólon
  Idiopático                             & 4                     & 1                     & 5                   & \textbf{4,9}     \\
  Fecaloma                                           & 3                     & 0                     & 3                   & \textbf{2,9}     \\
Fístula Perianal                                   & 3                     & 0                     & 3                   & \textbf{2,9}     \\
Lipidose
  Hepática                                & 0                     & 3                     & 3                   & \textbf{2,9}     \\
Dilatação
  Torção Gástrica                        & 2                     & 0                     & 2                   & \textbf{1,9}     \\
Mucocelo                                           & 2                     & 0                     & 2                   & \textbf{1,9}     \\
Peritonite Sética                                  & 0                     & 2                     & 2                   & \textbf{1,9}     \\
Triadite Felina                                    & 0                     & 2                     & 2                   & \textbf{1,9}     \\
Cirrose
  Hepática                                 & 1                     & 0                     & 1                   & \textbf{1,0}     \\
Colecistite                                        & 1                     & 0                     & 1                   & \textbf{1,0}     \\
Sialoadenose                                       & 1                     & 0                     & 1                   & \textbf{1,0}     \\
\textbf{TOTAL}                                     & 60                    & 43                    & 103                 & \textbf{100}     
\end{tblr}
\end{table}

\subsubsection{Nefrologia e Urologia}

Estas duas especialidades- Nefrologia e Urologia- dedicam-se ao estudo e tratamento de doenças que afetam o sistema urinário, constituído pelos rins, ureteres, bexiga e uretra.
Esta área foi, a par da a Gastroenterologia e Glândulas Anexas, uma das que apresentou maior casuística, com um total de $n=71$ casos registados, o que corresponde a uma frequência relativa de $14,4\%$ do total de casos da Clínica Médica. Destaca-se ainda por ter sido uma das áreas em que o número de casos em felídeos foi significativamente superior ao observado em canídeos.

Dado o elevado número de casos, a Nefrologia e Urologia assumiram particular relevância para o autor, permitindo consolidar os conhecimentos previamente adquiridos ao longo da formação universitária, tanto no que respeita às abordagens clínicas e terapêuticas, como à realização e interpretação de exames complementares de diagnóstico.Entre estes, salientam-se a Urinálise (tipos I e II), a Urocultura, as análises bioquímicas e mensuração da densidade urinária.

De acordo com a tabela \ref{tab:t5}, conclui-se que a doença com maior representatividade, com larga margem, foi a Doença Renal Crónica, contabilizando $n=37$ casos ($Fr=52,1\%$). Apenas duas doenças apresentaram maior incidência em cães do que em gatos: a Hidronefrose e Cistolitíase. Importa ainda referir que, dentro desta área, ocorreram situações que implicaram intervenção cirúrgica e que, por esse motivo, foram contabilizadas na secção de Clínica Cirúrgica, como os casos de Ureterolitíase.

% \usepackage{tabularray}
\begin{table}[h!]
\caption{ Distribuição da casuística recolhida na especialidade Nefrologia e Urologia, por espécie animal 
(Fip), por frequência absoluta (Fi), e frequência relativa em percentagem (Fr (\%))} 
\label{tab:t5}
\centering
\begin{tblr}{
  width = \linewidth,
  colspec = {Q[408]Q[160]Q[146]Q[131]Q[92]},
  cells = {c},
  hlines,
  vlines,
}
\textbf{Nefrologia e urologia}     & \textbf{Fi (Canina)} & \textbf{Fi (Felina)} & \textbf{Fi (Total)} & \textbf{Fr (\%)} \\
Doença Renal Crónica          & 7                    & 30                   & 37                  & \textbf{52,1}    \\
Cistite Idiopática Felina          & 0                    & 7                    & 7                   & \textbf{9,9}     \\
Cistolitiase                       & 4                    & 2                    & 6                   & \textbf{8,5}     \\
Ureterolitíase                     & 0                    & 6                    & 6                   & \textbf{8,5}     \\
Uretrolitíase                      & 0                    & 5                    & 5                   & \textbf{7,0}     \\
Infeção do Trato Urinário Inferior & 0                    & 3                    & 3                   & \textbf{4,2}     \\
Doença Renal Poliquística          & 0                    & 2                    & 2                   & \textbf{2,8}     \\
Hidronefrose                       & 2                    & 0                    & 2                   & \textbf{2,8}     \\
Lesão Renal Aguda   & 0                    & 1                    & 1                   & \textbf{1,4}     \\
Megaureter                         & 0                    & 1                    & 1                   & \textbf{1,4}     \\
Ureter Ectópico                    & 0                    & 1                    & 1                   & \textbf{1,4}     \\
\textbf{TOTAL}                     & 13                   & 58                   & 71                  & \textbf{100}     
\end{tblr}
\end{table}

\subsubsection{Cardiologia}

A especialidade de Cardiologia recai sobre as afeções do coração e sistema cardiovascular. Durante o seu estágio o aluno contactou com $n=69$ casos.

De acordo com tabela \ref{tab:t6}, dos $69$ casos, as afeções mais observadas do foro cardiológico foram a Doença Degenerativa da Válvula Mitral em cães e a Cardiomiopatia Hipertrófica em gatos, com uma frequência relativa de $39,1\%$ e $17,4\%$, respetivamente.

% \usepackage{tabularray}
\begin{table}[h!]
\caption{Distribuição da casuística recolhida na especialidade de Cardiologia, por espécie animal (Fip), 
por frequência absoluta (Fi), e frequência relativa em percentagem (Fr (\%)) } 
\label{tab:t6}
\centering
\begin{tblr}{
  width = \linewidth,
  colspec = {Q[492]Q[133]Q[121]Q[108]Q[83]},
  cells = {c},
  hlines,
  vlines,
}
\textbf{Cardiologia}                         & \textbf{Fi (Canina)} & \textbf{Fi (Felina)} & \textbf{Fi (Total)} & \textbf{Fr (\%)} \\
Doença Degenerativa da Válvula Mitral  & 27                   & 0                    & 27                  & \textbf{39,1}    \\
Cardiomiopatia Hipertrófica                  & 0                    & 12                   & 12                  & \textbf{17,4}    \\
Cardiomiopatia Dilatada                      & 9                    & 0                    & 9                   & \textbf{13,0}    \\
Bloqueio Atrioventricular                    & 3                    & 0                    & 3                   & \textbf{4,3}     \\
Tromboembolismo Aórtico                      & 0                    & 3                    & 3                   & \textbf{4,3}     \\
Estenose Sub-Aórtica                         & 3                    & 0                    & 3                   & \textbf{4,3}     \\
Falso Tendão                                 & 3                    & 0                    & 3                   & \textbf{4,3}     \\
Movimento Anterior Sistólico            & 0                    & 3                    & 3                   & \textbf{4,3}     \\
Persistência Ducto Arterioso                 & 2                    & 0                    & 2                   & \textbf{2,9}     \\
Estenose Pulmonar                            & 0                    & 1                    & 1                   & \textbf{1,4}     \\
Insuficiência da Válvula Tricúspide          & 1                    & 0                    & 1                   & \textbf{1,4}     \\
Comunicação Interventricular                 & 0                    & 1                    & 1                   & \textbf{1,4}     \\
Cor Triatriatum                               & 1                    & 0                    & 1                   & \textbf{1,4}     \\
\textbf{TOTAL}                               & 49                   & 20                   & 69                  & \textbf{100}     
\end{tblr}
\end{table}

\subsubsection{Oncologia}

A especialidade de Oncologia dedica-se ao estudo de todos os processos neoplásicos, desde o seu diagnóstico, tratamento e monitorização da evolução da doença. Nesta área foram observados um total de $n=53$ casos, correspondendo a $Fr=10,7\%$ da casuística total observada na Clínica Médica.

De acordo com a tabela \ref{tab:t7}, podemos aferir que a afeção com maior representatividade é o Linfoma Alimentar, com $n=15$ casos recolhidos ($Fr=28,3\%$). De salientar que, nesta afeção, a prevalência foi observada unicamente em gatos. 

% \usepackage{tabularray}
\begin{table}[h!]
\caption{Distribuição da casuística recolhida na especialidade de Oncologia, por espécie animal (Fip), 
por frequência absoluta (Fi), e frequência relativa em percentagem (Fr (\%)) } 
\label{tab:t7}
\centering
\begin{tblr}{
  width = \linewidth,
  colspec = {Q[469]Q[144]Q[135]Q[108]Q[83]},
  cells = {c},
  hlines,
  vlines,
}
\textbf{Oncologia}                             & \textbf{Fip (Canina)} & \textbf{Fip (Felina)} & \textbf{Fi (Total)} & \textbf{Fr (\%)} \\
Linfoma
  Alimentar                            & 0                     & 15                    & 15                  & \textbf{28,3}    \\
Carcinoma
  das células Escamosas              & 0                     & 7                     & 7                   & \textbf{13,2}    \\
Hemangiossarcoma
  Esplénico                   & 6                     & 0                     & 6                   & \textbf{11,3}    \\
Mastocitoma
  Cutâneo                          & 6                     & 0                     & 6                   & \textbf{11,3}    \\
Carcinoma
  Mamário                            & 1                     & 4                     & 5                   & \textbf{9,4}     \\
Carcinoma
  das Células de Transição da Bexiga & 2                     & 0                     & 2                   & \textbf{3,8}     \\
Insulinoma                                     & 2                     & 0                     & 2                   & \textbf{3,8}     \\
Linfoma
  Mediastínico                         & 0                     & 2                     & 2                   & \textbf{3,8}     \\
Melanoma
  oral                                & 2                     & 0                     & 2                   & \textbf{3,8}     \\
Adenocarcinoma
  das Glândulas Hepatóides      & 1                     & 0                     & 1                   & \textbf{1,9}     \\
Adenoma
  Esplénico                            & 1                     & 0                     & 1                   & \textbf{1,9}     \\
Adenoma
  Hepático                             & 1                     & 0                     & 1                   & \textbf{1,9}     \\
Carcinoma Pulmão                               & 1                     & 0                     & 1                   & \textbf{1,9}     \\
Hemangiossarcoma
  Cardíaco                    & 1                     & 0                     & 1                   & \textbf{1,9}     \\
Hepatocarcinoma                                & 1                     & 0                     & 1                   & \textbf{1,9}     \\
Leucemia                                       & 1                     & 0                     & 1                   & \textbf{1,9}     \\
Linfoma
  Gástrico                             & 1                     & 0                     & 1                   & \textbf{1,9}     \\
Linfoma Ocular                                 & 0                     & 1                     & 1                   & \textbf{1,9}     \\
Melanocitoma
  Palpebral                       & 1                     & 0                     & 1                   & \textbf{1,9}     \\
Osteossarcoma                                  & 1                     & 0                     & 1                   & \textbf{1,9}     \\
Timoma                                         & 1                     & 0                     & 1                   & \textbf{1,9}     \\
\textbf{TOTAL}                                 & 24                    & 29                    & 53                  & \textbf{ 100 }   
\end{tblr}
\end{table}

\subsubsection{Doenças Infeciosas e Parasitárias}

A especialidade de Doenças Infeciosas e Parasitárias representou $7,7\%$ da totalidade da casuística acompanhada em clínica médica, correspondendo a um total de $n=38$ casos. 

Durante o estágio, o autor teve a oportunidade de realizar diversos testes rápidos de diagnóstico, nomeadamente para FIV, FELV, Coronavírus e Parvovírus, o que permitiu o desenvolvimento de competências práticas fundamentais.

 De acordo com a tabela \ref{tab:t8}, a afeção mais prevalente foi a Parvovirose Canina/Panleucopénia Felina, com $n=11$ casos ($Fr=28,9\%$).

Importa ainda referir que, ao longo do estágio, foram observados mais casos desta especialidade na espécie felina.

% \usepackage{tabularray}
\begin{table}
\caption{Distribuição da casuística recolhida na especialidade de Doenças infeciosas e parasitárias, por espécie animal (Fip), 
por frequência absoluta (Fi), e frequência relativa em percentagem (Fr (\%)) } 
\label{tab:t8}
\centering
\begin{tblr}{
  width = \linewidth,
  colspec = {Q[465]Q[140]Q[131]Q[115]Q[85]},
  cells = {c},
  hlines,
  vlines,
}
\textbf{Doenças infeciosas e parasitárias} & \textbf{Fi (Canina)} & \textbf{Fi (Felina)} & \textbf{Fi (Total)} & \textbf{Fr (\%)} \\
Parvovirose Canina/ Panleucopenia Felina   & 5                    & 6                    & 11                  & \textbf{28,9}    \\
Leucemia Felina (FELV)                     & 0                    & 5                    & 5                   & \textbf{13,2}    \\
Imunodeficiência Viral Felina (FIV)        & 0                    & 4                    & 4                   & \textbf{10,5}    \\
Leishmaniose                               & 4                    & 0                    & 4                   & \textbf{10,5}    \\
Peritonite Infeciosa Felina                & 0                    & 4                    & 4                   & \textbf{10,5}    \\
Calicivirose                               & 0                    & 3                    & 3                   & \textbf{7,9}    \\
Leptospirose                               & 3                    & 0                    & 3                   & \textbf{7,9}     \\
Coriza                                     & 0                    & 2                    & 2                   & \textbf{5,3}     \\
Coronavírus                                & 1                    & 0                    & 1                   & \textbf{2,6}     \\
Dirofilariose                              & 1                    & 0                    & 1                   & \textbf{2,6}     \\
\textbf{TOTAL}                             & 14                   & 24                   & 38                  & \textbf{100}     
\end{tblr}
\end{table}

\subsubsection{Endocrinologia}

Na especialidade Endocrinologia, área que se dedica ao estudo e tratamento de disfunções das glândulas endócrinas. Foram observados um total de $n=33$ casos ($Fr=6,7\%$). Durante o acompanhamento dos casos nesta especialidade, o estagiário assistiu à abordagem clínica dos médicos veterinários na realização do diagnóstico da doença endócrina e na apresentação das opções de tratamento e respetiva monitorização.
Durante o acompanhamento dos casos nesta especialidade, o estagiário assistiu à abordagem clínica dos médicos veterinários na realização do diagnóstico da doença endócrina e na apresentação das opções de tratamento e respetiva monitorização.

Através da tabela \ref{tab:t9}, verifica-se que a afeção Diabetes \textit{mellitus} foi a que obteve maior representatividade, com um total de $n=14$ casos. Estes catorze casos ficaram divididos em quatro casos da espécie canina e dez da espécie felina, obtendo uma $Fr=42,4\%$ dentro da especialidade. Sendo seguido pela afeção Hipertiroidismo, que obteve também uma elevada representatividade com uma $Fr=24,2\%$. De salientar que foram recolhidos mais casos em gatos que cães nesta especialidade.

% \usepackage{tabularray}
\begin{table}
\caption{Distribuição da casuística recolhida na especialidade de Endocrinologia, por espécie animal (Fip), 
por frequência absoluta (Fi), e frequência relativa em percentagem (Fr (\%)) } 
\label{tab:t9}
\centering
\begin{tblr}{
  width = \linewidth,
  colspec = {Q[475]Q[138]Q[127]Q[113]Q[85]},
  cells = {c},
  hlines,
  vlines,
}
\textbf{Endocrinologia}                   & \textbf{Fi (Canina)} & \textbf{Fi (Felina)} & \textbf{Fi (Total)} & \textbf{Fr (\%)} \\
Diabetes \textit{mellitus}                & 4                    & 10                   & 14                  & \textbf{42,4}    \\
Hipertiroidismo                           & 0                    & 8                    & 8                   & \textbf{24,2}    \\
Hiperadrenocorticismo Hipófise-Dependente & 4                    & 0                    & 4                   & \textbf{12,1}    \\
Hipoadrenocorticismo                      & 3                    & 0                    & 3                   & \textbf{9,1}     \\
Hipotiroidismo                            & 3                    & 0                    & 3                   & \textbf{9,1}     \\
Diabetes \textit{insipidus}                        & 1                    & 0                    & 1                   & \textbf{3,0}     \\
\textbf{TOTAL}                            & 15                   & 18                   & 33                  & \textbf{100}     
\end{tblr}
\end{table}

\subsubsection{Dermatologia}

A especialidade de Dermatologia dedica-se ao estudo, diagnóstico e tratamento das afeções relacionadas com a pele. Durante o estágio, foram registados um total de $n=25$ casos nesta área, sendo os cães os doentes com maior representatividade, sobretudo devido ao elevado número de diagnósticos de Dermatite Atópica observados no HVP.

A Dermatite Atópica/Síndrome Atópica Felina foi a doença mais frequentemente identificada, representando um total de $n=17$ casos ($Fr=68,0\%$), dos quais 15 ocorreram em cães. 

A experiência adquirida nesta especialidade permitiu o desenvolvimento e consolidação de competências práticas, nomeadamente na colheita de lesões cutâneas através de técnicas como a recolha de amostras por aposição, teste de fita-cola e tricograma. Além disso, foi possível realizar a coloração adequada das amostras e proceder à sua interpretação microscópica, contribuindo significativamente para o diagnóstico das dermopatias.

% \usepackage{tabularray}
\begin{table}[h!]
\caption{Distribuição da casuística recolhida na especialidade de Dermatologia, por espécie animal (Fip), 
por frequência absoluta (Fi), e frequência relativa em percentagem (Fr (\%)) } 
\label{tab:t10}
\centering
\begin{tblr}{
  width = \linewidth,
  colspec = {Q[471]Q[140]Q[129]Q[113]Q[85]},
  cells = {c},
  hlines,
  vlines,
}
\textbf{Dermatologia }                     & \textbf{Fi (Canina)} & \textbf{Fi (Felina)} & \textbf{Fi (Total)} & \textbf{Fr (\%)} \\
Dermatite atópica/Síndrome Atópica Felina & 15                   & 2                    & 17                  & \textbf{68,0}    \\
Alopécia X                                 & 3                    & 0                    & 3                   & \textbf{12,0}    \\
Pioderma Superficial                       & 3                    & 0                    & 3                   & \textbf{12,0}    \\
Pênfigo Foliáceo                           & 1                    & 1                    & 2                   & \textbf{8,0}     \\
\textbf{TOTAL}                             & 22                   & 3                    & 25                  & \textbf{100}     
\end{tblr}
\end{table}
\subsubsection{Neurologia}

A especialidade de Neurologia dedica-se ao estudo, diagnóstico e terapêutica de doenças do sistema nervoso, sendo necessário e de extrema importância saber realizar um exame neurológico pormenorizado e ter conhecimento da neuroanatomia, bem como saber interpretar exames imagiológicos, sendo de particular interesse nesta área, a Tomografia Computorizada e a Radiografia. Não menos importante, saber fazer um bom exame ortopédico, é muitas vezes benéfico na diferenciação de lesões musculo-esqueléticas de lesões neurológicas.

A neurologia representou $4,7\%$ dos casos registados em Clínica Médica e, observando a tabela \ref{tab:t11} , é possível aferir, que os cães representaram a totalidade dos casos ($n=22$)
Salienta-se que a Hérnia Discal obteve $n=8$ casos ($Fr=34,8\%$) da casuística na especialidade de Neurologia. 

Alguns casos tiveram seguimento para Cirurgia, vindo a ser contabilizados na Clínica Cirúrgica, como é o caso das Hérnias Discais, onde foram posteriormente realizados procedimentos como Hemilaminectomia e \textit{Ventral Slot}.

% \usepackage{tabularray}
\begin{table}[h!]
\caption{Distribuição da casuística recolhida na especialidade de Neurologia, por espécie animal (Fip), 
por frequência absoluta (Fi), e frequência relativa em percentagem (Fr (\%)) } 
\label{tab:t11}
\centering
\begin{tblr}{
  width = \linewidth,
  colspec = {Q[396]Q[162]Q[148]Q[133]Q[94]},
  cells = {c},
  hlines,
  vlines,
}
\textbf{Neurologia}            & \textbf{Fi (Canina)} & \textbf{Fi (Felina)} & \textbf{Fi (Total)} & \textbf{Fr (\%)} \\
Hérnia Discal                  & 8                    & 0                    & 8                   & \textbf{34,8}    \\
Epilepsia Idiopática           & 5                    & 0                    & 5                   & \textbf{21,7}    \\
Meningoencefalite              & 5                    & 0                    & 5                   & \textbf{21,7}    \\
Síndrome Vestibular Geriátrico & 2                    & 0                    & 2                   & \textbf{8,7}     \\
Discoespondilite               & 1                    & 0                    & 1                   & \textbf{4,3}     \\
Malformação \textit{Chiari-Like}        & 1                    & 0                    & 1                   & \textbf{4,3}     \\
Miastenia \textit{gravis}               & 1                    & 0                    & 1                   & \textbf{4,3}     \\
\textbf{TOTAL}                 & 23                   & 0                    & 23                  & \textbf{100}     
\end{tblr}
\end{table}

\subsubsection{Pneumologia}

A Pneumologia é a especialidade que se dedica ao estudo e tratamento de doenças pertencentes ao trato respiratório inferior, englobando a Traqueia, Brônquios, Pulmões e Pleura. Esta área foi responsável por $4,3\%$ da totalidade da casuística recolhida em Clínica Médica. 
Durante este estágio, foi possível ao estagiário assistir à realização e interpretação de radiografias torácicas, fundamentais para o diagnóstico das principais afeções respiratórias.
De acordo com a tabela \ref{tab:t12} , é possível aferir que a doença com maior representatividade foi o Colapso Traqueal, com uma frequência absoluta de $n=6$ casos ($Fr=28,6\%$).
% \usepackage{tabularray}
\begin{table}[h!]
\caption{Distribuição da casuística recolhida na especialidade de Pneumologia, por espécie animal (Fip), 
por frequência absoluta (Fi), e frequência relativa em percentagem (Fr (\%)) } 
\label{tab:t12}
\centering
\begin{tblr}{
  width = \linewidth,
  colspec = {Q[462]Q[142]Q[133]Q[115]Q[85]},
  cells = {c},
  hlines,
  vlines,
}
\textbf{Pneumologia}                  & \textbf{Fi (Canina)} & \textbf{Fi (Felina)} & \textbf{Fi (Total)} & \textbf{Fr (\%)} \\
Colapso Traqueal                      & 6                    & 0            & 6                   & \textbf{28,6}    \\
Pneumonia Aspirativa                  & 4                    & 0           & 4                   & \textbf{19,0}    \\
Asma Felina                           & 0                    & 3                    & 3                   & \textbf{14,3}    \\
Bronquite Crónica                     & 3                    & 0                    & 3                   & \textbf{14,3}    \\
Contusão Pulmonar secundária a Trauma & 2                    & 0                    & 2                   & \textbf{9,5}     \\
Quilotórax Idiopático                 & 0                    & 2                    & 2                   & \textbf{9,5}     \\
Hérnia Diafragmática Traumática                  & 1                    & 0                    & 1                   & \textbf{4,8}     \\
\textbf{TOTAL}                        & 16                   & 5                    & 21                  & \textbf{100}     
\end{tblr}
\end{table}

\subsubsection{Otorrinolaringologia}
A especialidade de Otorrinolaringologia dedica-se ao estudo, diagnóstico e tratamento das doenças que afetam os Ouvidos, Nariz, Seios Paranasais, Faringe e Laringe.

Esta área foi responsável por $3,6\%$ da totalidade da casuística registada em Clínica Médica. De acordo com os dados apresentados na tabela \ref{tab:t13}, verifica-se que todos os casos incidiram sobre a região do Ouvido, com maior representatividade em cães do que em gatos. 

A afeção mais frequentemente diagnosticada foi a Otite Externa secundária a \textit{Malassezia spp.}, constituindo a principal causa de afeção otológica identificada durante o período de estágio.

% \usepackage{tabularray}
\begin{table}[h!]
\caption{Distribuição da casuística recolhida na especialidade de Otorrinolaringologia, por espécie animal (Fip), 
por frequência absoluta (Fi), e frequência relativa em percentagem (Fr (\%)) } 
\label{tab:t13}
\centering
\begin{tblr}{
  width = \linewidth,
  colspec = {Q[438]Q[150]Q[138]Q[123]Q[88]},
  cells = {c},
  hlines,
  vlines,
}
\textbf{Otorrinolaringologia}                       & \textbf{Fi (Canina)} & \textbf{Fi (Felina)} & \textbf{Fi (Total)} & \textbf{Fr (\%)} \\
Otite Externa por \textbf{\textit{Malassezia spp.}} & 7                    & 0                    & 7                   & \textbf{38,9}    \\
Otite Externa Bacteriana                            & 6                    & 0                    & 6                   & \textbf{33,3}    \\
Oto-Hematoma                                        & 2                    & 1                    & 3                   & \textbf{16,7}    \\
Otite Média Bacteriana                              & 1                    & 0                    & 1                   & \textbf{5,6}     \\
Otite Média por Pólipo Nasofaríngeo                 & 0                    & 1                    & 1                   & \textbf{5,6}     \\
\textbf{TOTAL}                                      & 16                   & 2                    & 18                  & \textbf{100}     
\end{tblr}
\end{table}

\subsubsection{Oftalmologia}

A especialidade de Oftalmologia dedica-se ao estudo, diagnóstico e tratamento de doenças oculares. Trata-se de uma área que requer a realização de exames complementares específicos, tais como o Teste de Schirmer, a Medição da Pressão Intraocular, o Teste de Fluoresceína, a Biomicroscopia com Lâmpada de Fenda e a Ultrassonografia Ocular.

Durante o estágio, o autor acompanhou $n=11$ casos desta especialidade durante o período reservado a consultas, o que, em parte, justifica a menor casuística registada nesta área.

De acordo com a tabela \ref{tab:t14}, as doenças com maior representatividade foram a Queratoconjuntivite Seca e a Úlcera Superficial da Córnea, ambas com uma frequência absoluta de dois casos ($Fr=18,2\%$)

Importa referir que todos os casos observados nesta área se restringiram à espécie canina.

% \usepackage{tabularray}
\begin{table}[h!]
\caption{Distribuição da casuística recolhida na especialidade de Oftalmologia, por espécie animal (Fip), 
por frequência absoluta (Fi), e frequência relativa em percentagem (Fr (\%)) } 
\label{tab:t14}
\centering
\begin{tblr}{
  width = \linewidth,
  colspec = {Q[527]Q[121]Q[112]Q[100]Q[79]},
  cells = {c},
  hlines,
  vlines,
}
\textbf{Oftalmologia}                                             & \textbf{Fi (Canina)} & \textbf{Fi (Felina)} & \textbf{Fi (Total)} & \textbf{Fr (\%)} \\
Queratoconjuntivite seca                                          & 2                    & 0                    & 2                   & \textbf{18,2}    \\
Úlcera superficial da córnea                                      & 2                    & 0                    & 2                   & \textbf{18,2}    \\
Cataratas                                                         & 1                    & 0                    & 1                   & \textbf{9,1}     \\
Descemetocélio                                                    & 1                    & 0                    & 1                   & \textbf{9,1}     \\
Entrópio                                                          & 1                    & 0                    & 1                   & \textbf{9,1}     \\
Ectrópio                                                          & 1                    & 0                    & 1                   & \textbf{9,1}     \\
Prolapso da glândula da membrana nictitante (\textit{cherry eye}) & 1                    & 0                    & 1                   & \textbf{9,1}     \\
Síndrome de Degeneração Retiniana Adquirida Súbita                & 1                    & 0                    & 1                   & \textbf{9,1}     \\
Uveíte                                                            & 1                    & 0                    & 1                   & \textbf{9,1}     \\
\textbf{TOTAL}                                                    & 11                   & 0                    & 11                  & \textbf{100}     
\end{tblr}
\end{table}

\subsubsection{Teriogenologia}

Teriogenologia é a área da Medicina Veterinária que se dedica ao estudo da fisiologia e das patologias do sistema reprodutivo animal. Foram recolhidos $n=10$ casos, sendo responsável por $2,0\%$ da totalidade da casuística registada em Clínica Médica.

De acordo com a Tabela \ref{tab:t15}, verifica-se que todos os casos se restringiram à espécie canina. 
A doença com maior representatividade foi a presença de Quistos Prostáticos, correspondendo a $40\%$ dos casos registados na área.
Tal como verificado noutras especialidades, os casos de Piómetra implicaram seguimento cirúrgico, sendo posteriormente contabilizados na categoria de Óvario-histerectomia, no âmbito da Clínica Cirúrgica.

% \usepackage{tabularray}
\begin{table}[h!]
\caption{Distribuição da casuística recolhida na especialidade de Teriogenologia, por espécie animal (Fip), 
por frequência absoluta (Fi), e frequência relativa em percentagem (Fr (\%)) } 
\label{tab:t15}
\centering
\begin{tblr}{
  width = \linewidth,
  colspec = {Q[275]Q[194]Q[179]Q[160]Q[113]},
  cells = {c},
  hlines,
  vlines,
}
\textbf{Teriogenologia} & \textbf{Fi (Canina)} & \textbf{Fi (Felina)} & \textbf{Fi (Total)} & \textbf{Fr (\%)} \\
Quisto Prostático       & 4                    & 0                    & 4                   & \textbf{40,0}    \\
Piómetra                & 3                    & 0                    & 3                   & \textbf{30,0}    \\
Parafimose              & 1                    & 0                    & 1                   & \textbf{10,0}    \\
Prostatite              & 1                    & 0                    & 1                   & \textbf{10,0}    \\
Vaginite                & 1                    & 0                    & 1                   & \textbf{10,0}    \\
\textbf{TOTAL}          & 10                   & 0                    & 10                  & \textbf{ 100 }   
\end{tblr}
\end{table}

\subsubsection{Traumatologia e Doenças Musculoesqueléticas}

Esta área dedica-se ao estudo, diagnóstico e tratamento das afeções do aparelho locomotor. A abordagem eficaz destas doenças exige um sólido conhecimento anatómico, bem como competências práticas na realização de exames ortopédicos e neurológicos. Para além disso, assume particular relevância a capacidade de interpretação de exames de imagiologia, como a Radiografia e a TC, fundamentais para uma avaliação completa das lesões musculoesqueléticas.

Durante o estágio, foram registados $n=8$ casos nesta especialidade, correspondendo a uma $Fr=1.6\%$ da casuística total observada em Clínica Médica.

A doença com maior representatividade foi a Rotura do Ligamento Cruzado Cranial, com $n=4$ casos ($Fr=50,0\%$)

\begin{table}[h!]
\caption{Distribuição da casuística recolhida na especialidade de Traumatologia e Doenças Musculoesqueléticas, por espécie animal (Fip), 
por frequência absoluta (Fi), e frequência relativa em percentagem (Fr (\%)) } 
\label{tab:t16}
\centering
\begin{tblr}{
  width = \linewidth,
  colspec = {Q[502]Q[129]Q[119]Q[106]Q[81]},
  cells = {c},
  hlines,
  vlines,
}
\textbf{Traumatologia e Doenças Musculoesqueléticas} & \textbf{Fi (Canina)} & \textbf{Fi (Felina)} & \textbf{Fi (Total)} & \textbf{Fr (\%)} \\
Rotura do Ligamento Cruzado Cranial                  & 4                    & 0                    & 4                   & \textbf{50,0}    \\
Fraturas Ósseas                                      & 1                    & 2                    & 3                   & \textbf{37,5}    \\
Poliartrite
  Imunomediada                           & 1                    & 0                    & 1                   & \textbf{12,5}    \\
\textbf{TOTAL}                                       & 6                    & 2                    & 8                   & \textbf{ 100 }             
\end{tblr}
\end{table}

\subsubsection{Hematologia}

A especialidade de Hematologia dedica-se ao estudo, diagnóstico e tratamento de doenças do foro sanguíneo. No decurso do estágio, foram registados $n=6$ casos nesta área, correspondendo a uma $Fr= 1,2\%$ do total de casos observados em Clínica Médica.

De acordo com a tabela \ref{tab:t17}, verifica-se que a doença com maior representatividade foi a Anemia Hemolítica Imunomediada.

% \usepackage{tabularray}
\begin{table}[h!]
\caption{Distribuição da casuística recolhida na especialidade de Hematologia, por espécie animal (Fip), 
por frequência absoluta (Fi), e frequência relativa em percentagem (Fr (\%)) } 
\label{tab:t17}
\centering
\begin{tblr}{
  width = \linewidth,
  colspec = {Q[425]Q[154]Q[142]Q[127]Q[90]},
  cells = {c},
  hlines,
  vlines,
}
\textbf{Hematologia}           & \textbf{Fi (Canina)} & \textbf{Fi (Felina)} & \textbf{Fi (Total)} & \textbf{Fr (\%)} \\
Anemia Hemolitica Imunomediada & 4                    & 1                    & 5                   & \textbf{83,3}    \\
Trombocitopenia Imunomediada   & 1                    & 0                    & 1                   & \textbf{16,7}    \\
\textbf{TOTAL}                 & 5                    & 1                    & 6                   & \textbf{100}     
\end{tblr}
\end{table}

\subsubsection{Toxicologia}

A especialidade de Toxicologia dedica-se ao estudo, diagnóstico e tratamento dos sinais clínicos provocados resultantes da exposição a substâncias tóxicas, sejam elas ingeridas, inaladas, absorvidas ou inoculadas. 
Durante o estágio, a abordagem e monitorização clínica dos doentes, bem como a realização de provas de coagulação, foram as principais atividades desenvolvidas no âmbito desta área, dada a relevância da avaliação da função hemostática em muitos quadros de intoxicação.

Esta especialidade apresentou um peso reduzido na casuística total da clínica médica, representando apenas $1,0\%$ dos casos registados.

 De acordo com a tabela \ref{tab:t18}, foram identifcados três tipos distintos de Intoxicação, sendo a intoxicação por Rodenticidas a mais prevalente, com $n=3$ casos ($Fr=60\%$).

\begin{table}[h!]
\caption{Distribuição da casuística recolhida na especialidade de Toxicologia, por espécie animal (Fip), 
por frequência absoluta (Fi), e frequência relativa em percentagem (Fr (\%)) } 
\label{tab:t18}
\centering
\begin{tblr}{
  width = \linewidth,
  colspec = {Q[431]Q[152]Q[140]Q[125]Q[88]},
  cells = {c},
  hlines,
  vlines,
}
\textbf{Toxicologia}                & \textbf{Fi (Canina)} & \textbf{Fi (Felina)} & \textbf{Fi (Total)} & \textbf{Fr (\%)} \\
Rodenticidas                        & 3                    & 0                    & 3                   & \textbf{60,0}    \\
Intoxicação por \textit{Thaumetopoea pityocampa} & 1                    & 0                    & 1                   & \textbf{20,0}    \\
Intoxicação por Paracetamol                         & 1                    & 0                    & 1                   & \textbf{20,0}    \\
\textbf{TOTAL}                      & 5                    & 0                    & 5                   & \textbf{100}             
\end{tblr}
\end{table}



\subsection{Clínica Cirúrgica}

Após a análise da casuística referente à Clínica Médica, constatou-se que a área da Clínica Cirúrgica apresentou um número inferior de casos, tendo sido registado um total de $n=60$ casos, o que corresponde a $10,2\%$ da casuística global observada.

A recolha da casuística foi realizada durante os turnos dedicados à Cirurgia, nos quais o autor acompanhou os doentes nas diferentes fases do procedimento cirúrgico: pré-operatório, intraoperatório e pós-operatório. 
Na fase pré-operatória, o autor colaborou com a equipa de anestesia na preparação dos doentes, participando no cálculo e administração da medicação pré-anestésica, no procedimento de venóclise, na administração de fluidoterapia, na tricotomia e assepsia da área a ser intervencionada, na entubação endotraqueal e na monitorização do doente até à entrada no bloco operatório. 

Durante a fase intraoperatória, o autor desempenhou a função de ajudante de cirurgião, o que lhe permitiu uma observação privilegiada e direta do procedimento cirúrgico realizado em cada doente. 

No período pós-operatório, a principal responsabilidade do estagiário consistiu na monitorização dos parâmetros vitais e recuperação dos doentes, assegurando a sua estabilidade.

Para facilitar a análise da casuística recolhida em Clínica Cirúrgica, os casos foram organizados em cinco categorias principais: Cirurgia de Tecidos Moles, Cirurgia Odontológica, Cirurgia Ortopédica, Neurocirurgia e Outros Procedimentos. 

De acordo com a tabela \ref{tab:t19}, que apresenta a distribuição da casuística cirúrgica, a Cirurgia de Tecidos Moles destacou-se como a categoria com maior número de casos registados, representando $76,7\%$ da totalidade da casuística cirúrgica.

% \usepackage{tabularray}
\begin{table}
\caption{Distribuição da casuística recolhida em Clínica Cirúrgica, expressa em Frequência 
absoluta por espécie animal (Fip), Frequência absoluta (Fi) e Frequência relativa em percentagem (Fr\%).  } 
\label{tab:t19}
\centering
\begin{tblr}{
  width = \linewidth,
  colspec = {Q[346]Q[183]Q[171]Q[135]Q[96]},
  cells = {c},
  hlines,
  vlines,
}
\textbf{Clínica Cirúrgica} & \textbf{Fip (Canina)} & \textbf{Fip (Felina)} & \textbf{Fi (Total)} & \textbf{Fr (\%)} \\
Cirurgia de Tecidos Moles  & 27                    & 19                    & 46                  & \textbf{76,7}    \\
Cirurgia Odontológica      & 9                     & 2                     & 11                  & \textbf{18,3}    \\
Neurocirurgia              & 2                     & 0                     & 2                   & \textbf{3,3}     \\
Cirurgia Ortopédica        & 0                     & 1                     & 1                   & \textbf{1,7}     \\
\textbf{TOTAL}             & 38                    & 22                    & 60                  & \textbf{100}     
\end{tblr}
\end{table}
\subsubsection{Cirurgia de Tecidos Moles}

No total, foram recolhidos $n=46$ casos de Cirurgia de Tecidos Moles. Como referido anteriormente, a Cirurgia de Tecidos Moles destacou-se como a categoria com maior número de casos registados, representando $76,7\%$ da totalidade da casuística cirúrgica.
De acordo com a tabela \ref{tab:t20}, os procedimentos de esterilização e castração, são os que obtiveram uma maior representatividade. A Óvario-Histerectomia representou $17,4\%$ dos casos, enquanto que, a Orquiectomia obteve uma frequência relativa de $21,7\%$.

% \usepackage{tabularray}
\begin{table}[h!]
\caption{Distribuição da casuística recolhida em Cirurgia de tecidos moles, expressa em Frequência 
absoluta por espécie animal (Fip), Frequência absoluta (Fi) e Frequência relativa em percentagem (Fr\%).  } 
\label{tab:t20}
\centering
\begin{tblr}{
  width = \linewidth,
  colspec = {Q[452]Q[150]Q[140]Q[112]Q[83]},
  cells = {c},
  hlines,
  vlines,
}
\textbf{Tecidos moles}                           & \textbf{Fip (Canina)} & \textbf{Fip (Felina)} & \textbf{Fi (Total)} & \textbf{Fr (\%)} \\
Orquiectomia                                     & 4                     & 6                     & 10                  & \textbf{21,7}    \\
Ovário-Histerectomia                       & 5                     & 3                     & 8                   & \textbf{17,4}    \\
Enterotomia                                      & 5                     & 1                     & 6                   & \textbf{13,0}    \\
Colocação de tubo de Esofagostomia               & 0                     & 4                     & 4                   & \textbf{8,7}     \\
Colocação de \textit{bypass} Ureteral Subcutâneo & 0                     & 3                     & 3                   & \textbf{6,5}     \\
Gastrotomia                                      & 3                     & 0                     & 3                   & \textbf{6,5}     \\
Mastectomia                                      & 2                     & 1                     & 3                   & \textbf{6,5}     \\
Nodulectomia                                     & 3                     & 0                     & 3                   & \textbf{6,5}     \\
Colocação de \textit{Pacemaker }                              & 2                     & 0                     & 2                   & \textbf{4,3}     \\
Resolução de Oto-Hematoma                        & 2                     & 0                     & 2                   & \textbf{4,3}     \\
Biópsia Pâncreas                                 & 1                     & 0                     & 1                   & \textbf{2,2}     \\
Enterectomia                                     & 0                     & 1                     & 1                   & \textbf{2,2}     \\
\textbf{TOTAL}                                   & 27                    & 19                    & 46                  & \textbf{100}     
\end{tblr}
\end{table}

\subsubsection{Cirurgia odontológica}

Nesta área foram recolhidos $n=11$ casos ($Fr=18,3\%$). De acordo com a tabela \ref{tab:t21}, podemos aferir que o procedimento odontológico com maior representatividade foi a Destartarização e/ou Extração Dentária, com uma frequência absoluta de $n=10$ casos ($Fr=90,9\%$).

% \usepackage{tabularray}
\begin{table}[h!]
\caption{Distribuição da casuística recolhida em Cirurgia odontológica, expressa em Frequência 
absoluta por espécie animal (Fip), Frequência absoluta (Fi) e Frequência relativa em percentagem (Fr\%).  } 
\label{tab:t21}
\centering
\begin{tblr}{
  width = \linewidth,
  colspec = {Q[454]Q[146]Q[135]Q[119]Q[87]},
  cells = {c},
  hlines,
  vlines,
}
\textbf{Odontologia}                   & \textbf{Fi (Canina)} & \textbf{Fi (Felina)} & \textbf{Fi (Total)} & \textbf{Fr (\%)} \\
Destartarização e/ou Extração Dentária & 8                    & 2                    & 10                  & \textbf{90,9}    \\
Remoção de Melanoma Oral               & 1                    & 0                    & 1                   & \textbf{9,1}     \\
\textbf{TOTAL}                         & 9                    & 2                    & 11                  & \textbf{100}     
\end{tblr}
\end{table}

\subsubsection{Neurocirurgia}

Na área de Neurocirurgia, foram recolhidos apenas dois casos ($Fr=3,3\%$).
De acordo com a tabela \ref{tab:t22}, é possível verificar que os procedimentos cirúrgicos realizados foram Hemilaminectomia e \textit{Ventral Slot}, para fim de correção de Hérnias Discais.

% \usepackage{tabularray}
\begin{table}[h!]
\caption{Distribuição da casuística recolhida em Neurocirurgia, expressa em Frequência 
absoluta por espécie animal (Fip), Frequência absoluta (Fi) e Frequência relativa em percentagem (Fr\%).  } 
\label{tab:t22}
\centering
\begin{tblr}{
  width = \linewidth,
  colspec = {Q[281]Q[202]Q[188]Q[150]Q[106]},
  cells = {c},
  hlines,
  vlines,
}
\textbf{Neurocirurgia} & \textbf{Fip (Canina)} & \textbf{Fip (Felina)} & \textbf{Fi (Total)} & \textbf{Fr (\%)} \\
Hemilaminectomia       & 1                     & 0                     & 1                   & \textbf{50,0}    \\
\textit{Ventral Slot}           & 1                     & 0                     & 1                   & \textbf{50,0}    \\
\textbf{ TOTAL  }      & 2                     & 0                     & 2                   & \textbf{100}     
\end{tblr}
\end{table}

\subsubsection{Cirurgia Ortopédica}

Ao longo dos 6 meses de estágio, o autor teve oportunidade de assistir e auxiliar apenas em uma cirurgia ortopédica, a qual representou uma frequência relativa de $1,7\%$ dos casos registados na Clínica Cirúrgica. 

Importa, no entanto, salientar que o HVP dispõe de todos os meios necessários para a realização de cirurgias ortopédicas, sendo esta limitação casuística apenas reflexo da ausência de casos desta natureza durante o período em que o autor esteve integrado na equipa cirúrgica.

\begin{table}[h!]
\caption{Distribuição da casuística recolhida em Cirurgia Ortopédica, expressa em Frequência 
absoluta por espécie animal (Fip), Frequência absoluta (Fi) e Frequência relativa em percentagem (Fr\%).  } 
\label{tab:t23}
\centering
\begin{tblr}{
  width = \linewidth,
  colspec = {Q[488]Q[138]Q[129]Q[104]Q[81]},
  cells = {c},
  hlines,
  vlines,
}
\textbf{Cirurgia Ortopédica~}                        & \textbf{Fip (Canina)} & \textbf{Fip (Felina)} & \textbf{Fi (Total)} & \textbf{Fr (\%)} \\
osteossíntese de falanges com pinos & 0                     & 1                     & 1                   & \textbf{ 100 }   \\
\textbf{ TOTAL }                                     & 0                     & 1                     & 1                   & 100              
\end{tblr}
\end{table}


\subsection{Procedimentos Médicos}

Na tabela \ref{tab:t24}, estão representados alguns procedimentos que o autor realizou, auxiliou ou assistiu e que, não podem ser contabilizados nas áreas mencionadas previamente.
Nesta área destaca-se o procedimento, Cistocentese, com uma frequência absoluta de $n=77$ casos ($Fr=50,0\%$).

\begin{table}[h!]
\caption{Distribuição da casuística recolhida em Procedimentos Médicos, expressa em Frequência 
absoluta por espécie animal (Fip), Frequência absoluta (Fi) e Frequência relativa em percentagem (Fr\%).  } 
\label{tab:t24}
\centering
\begin{tblr}{
  width = \linewidth,
  colspec = {Q[348]Q[175]Q[162]Q[142]Q[102]},
  cells = {c},
  hlines,
  vlines,
}
\textbf{Outros procedimentos} & \textbf{Fi (Canina)} & \textbf{Fi (Felina)} & \textbf{Fi (Total)} & \textbf{Fr (\%)} \\
Cistocentese                  & 31                   & 46                   & 77                  & \textbf{50,0}    \\
Toracocentese                 & 13                   & 11                   & 24                  & \textbf{15,6}    \\
Limpeza de Feridas            & 11                   & 7                    & 18                  & \textbf{11,7}    \\
Transfusão Sanguínea          & 5                    & 6                    & ~11                 & \textbf{7,1}     \\
Eutanásia                     & 6                    & 4                    & 10                  & \textbf{6,5}     \\
Quimioterapia                 & 4                    & 6                    & 10                  & \textbf{6,5}     \\
Abdominocentese               & 4                    & 0                    & 4~~                 & \textbf{2,6}     \\
\textbf{TOTAL}                & 74                   & 80                   & 154                 & \textbf{100}              
\end{tblr}
\end{table}

\subsection{Exames Complementares de Diagnóstico}

Na tabela \ref{tab:t25}, podemos observar a casuística dos inúmeros exames complementares de diagnóstico em que o aluno participou. Como é possível aferir, os exames com maior representatividade foram, dentro da área de Imagiologia, a Ultrassonografia e Radiografia.

\begin{table}[h!]
\caption{Distribuição da casuística recolhida em Exames Complementares de Diagnóstico, expressa em Frequência 
absoluta por espécie animal (Fip), Frequência absoluta (Fi) e Frequência relativa em percentagem (Fr\%).  } 
\label{tab:t25}
\centering
\begin{tblr}{
  width = \linewidth,
  colspec = {Q[121]Q[165]Q[229]Q[123]Q[113]Q[102]Q[79]},
  cells = {c},
  cell{1}{1} = {c=3}{0.515\linewidth},
  cell{2}{1} = {r=5}{},
  cell{2}{2} = {r=3}{},
  cell{5}{2} = {c=2}{0.394\linewidth},
  cell{6}{2} = {c=2}{0.394\linewidth},
  cell{7}{1} = {c=3}{0.515\linewidth},
  cell{8}{1} = {c=3}{0.515\linewidth},
  cell{9}{1} = {c=3}{0.515\linewidth},
  cell{10}{1} = {c=3}{0.515\linewidth},
  cell{11}{1} = {c=3}{0.515\linewidth},
  cell{12}{1} = {c=2,r=3}{0.286\linewidth},
  cell{15}{1} = {c=2,r=2}{0.286\linewidth},
  cell{17}{1} = {c=3}{0.515\linewidth},
  vlines,
  hline{1-2,7-12,15,17-18} = {-}{},
  hline{3-4,13-14,16} = {3-7}{},
  hline{5-6} = {2-7}{},
}
\textbf{Exames Complementares de Diagnóstico} &                  &                        & \textbf{Fi (Canina)} & \textbf{Fi (Felina)} & \textbf{Fi (Total)} & \textbf{Fr (\%)} \\
Imagiolo- gia                                  & Ultrasso- nografia & Abdominal              & 156                  & 187                  & 343                 & \textbf{52,3}    \\
                                             &                  & Ecocardiografia        & 44                   & 27                   & 71                  & \textbf{10,8}    \\
                                             &                  & Ocular                 & 2                    & 0                    & 2                   & \textbf{0,3}     \\
                                             & Radiografia      &                        & 41                   & 35                   & 76                  & \textbf{11,6}    \\
                                             & Tomografia Computorizada               &                        & 26                   & 5                    & 29                  & \textbf{4,4}     \\
Eletrocardiografia                           &                  &                        & 15                   & 12                   & 27                  & \textbf{4,1}     \\
Colheita de líquido Cefalorraquidiano        &                  &                        & 4                    & 2                    & 6                   & \textbf{0,9}     \\
Urinálise                                    &                  &                        & 16                   & 25                   & 41                  & \textbf{6,3}     \\
Teste de Schirmer                            &                  &                        & 2                    & 0                    & 2                   & \textbf{0,3}     \\
Teste de Fluoresceína                        &                  &                        & 1                    & 0                    & 1                   & \textbf{0,2}     \\
Testes Rápidos                               &                  & Parvovirus/ Coronavirus & 7                    & 5                    & 12                  & \textbf{1,8}     \\
                                             &                  & SNAP PLi               & 1                    & 3                    & 4                   & \textbf{0,6}     \\
                                             &                  & FIV/FELV               & 0                    & 4                    & 4                   & \textbf{0,6}     \\
Anatomo-istopatologia                       &                  & Citologia              & 11                   & 18                   & 29                  & \textbf{4,4}     \\
                                             &                  & Biópsia                & 4                    & 3                    & 7                   & \textbf{1,1}     \\
\textbf{TOTAL}
                             &                  &                        & 330                  & 326                  & 656                 & 100              
\end{tblr}
\end{table}





